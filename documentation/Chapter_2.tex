%% Chapter-1.tex
%% Mac Radigan
%
%% Examples from SICP Chapter 2

    \section{Building Abstractions with Data}
        \subsection{Introduction to Data Abstraction}
            \subsubsection{Example: Arithmetic Operations for Rational Numbers}
Exercise 2.1. Define a better version of make-rat that handles both positive and negative arguments. Make-rat should normalize the sign so that if the rational number is positive, both the numerator and denominator are positive, and if the rational number is negative, only the numerator is negative.
              \schemelist{../chapter-2/sicp_ch2_e2-1.scm}
              \outlist{../output/sicp_ch2_e2-1.out}
            \subsubsection{Abstraction Barriers}

Exercise 2.2. Consider the problem of representing line segments in a plane. Each segment is represented as a pair of points: a starting point and an ending point. Define a constructor make-segment and selectors start-segment and end-segment that define the representation of segments in terms of points. Furthermore, a point can be represented as a pair of numbers: the x coordinate and the y coordinate. Accordingly, specify a constructor make-point and selectors x-point and y-point that define this representation. Finally, using your selectors and constructors, define a procedure midpoint-segment that takes a line segment as argument and returns its midpoint (the point whose coordinates are the average of the coordinates of the endpoints).  To try your procedures, you'll need a way to print points:

              \schemelist{../chapter-2/sicp_ch2_e2-2.scm}
              \outlist{../output/sicp_ch2_e2-2.out}

Exercise 2.3. Implement a representation for rectangles in a plane. (Hint: You may want to make use of exercise 2.2.) In terms of your constructors and selectors, create procedures that compute the perimeter and the area of a given rectangle. Now implement a different representation for rectangles.  Can you design your system with suitable abstraction barriers, so that the same perimeter and area procedures will work using either representation?

              \schemelist{../chapter-2/sicp_ch2_e2-3.scm}
              \outlist{../output/sicp_ch2_e2-3.out}
            \subsubsection{What Is Meant by Data?}

Exercise 2.4. Here is an alternative procedural representation of pairs. For this representation, verify that (car (cons x y)) yields x for any objects x and y.  What is the corresponding definition of cdr? (Hint: To verify that this works, make use of the substitution model of section 1.1.5.) 

\begin{equation}
cons\left(x,y\right) \triangleq \lambda m . m x y
\label{eq:church_pair}
\end{equation}
\newline

\begin{equation}
\begin{split}
car\left(z\right) & \triangleq \lambda z . z \lambda p q . p \\
& \to_\beta \lfloor{ ^{cons\left(x,y\right)} / _{z} \rfloor} car\left(z\right)  \\
& = \left( \lambda m . m x y \right) \lambda p q . p \\
& \to_\beta \left( \lambda p q . p \right) x y  \\
& \to_\beta x
\end{split}
\label{eq:church_car}
\end{equation}
\newline

\begin{equation}
\begin{split}
cdr\left(z\right) & \triangleq \lambda z . z \lambda p q . q \\
& \to_\beta \lfloor{ ^{cons\left(x,y\right)} / _{z} \rfloor} cdr\left(z\right)  \\
& = \left( \lambda m . m x y \right) \lambda p q . q \\
& \to_\beta \left( \lambda p q . q \right) x y  \\
& \to_\beta y
\end{split}
\label{eq:church_cdr}
\end{equation}
\newline
% 
              \schemelist{../chapter-2/sicp_ch2_e2-4.scm}
              \outlist{../output/sicp_ch2_e2-4.out}

            \subsubsection{Extended Exercise: Interval Arithmetic}
        \subsection{Hierarchical Data and the Closure Property}
Exercise 2.17.  Define a procedure last-pair that returns the list that contains only the last element of a given (nonempty) list:
              \schemelist{../chapter-2/sicp_ch2_e2-17.scm}
              \outlist{../output/sicp_ch2_e2-17.out}
Exercise 2.18.  Define a procedure reverse that takes a list as argument and returns a list of the same elements in reverse order:
              \schemelist{../chapter-2/sicp_ch2_e2-18.scm}
              \outlist{../output/sicp_ch2_e2-18.out}
            \subsubsection{Representing Sequences}
            \subsubsection{Hierarchical Structures}

Exercise 2.21. The procedure square-list takes a list of numbers as argument and returns a list of the squares of those numbers.

\begin{verbatim}
  (square-list (list 1 2 3 4))
    (1 4 9 16)
\end{verbatim}

Here are two different definitions of square-list. Complete both of them by filling in the missing expressions:

\begin{verbatim}
  (define (square-list items)
    (if (null? items)
      nil
    (cons <??> <??>)))

  (define (square-list items)
    (map <??> <??>))
\end{verbatim}

\begin{equation}
{\ell}^2\left(\mbox{items}\right) \triangleq
\begin{cases}
\mbox{nil} & \mbox{, null? items} \\
\mbox{items}_A^2 \mbox{ :: } {\ell}^2\left( \mbox{items}_D \right) & \mbox{ o.w.}
\end{cases}
\caption{{\ell}^2 using recursion}
\label{eq:l2_map}
\end{equation}
\newline

\begin{equation}
{\ell}^2\left(\mbox{items}\right) \triangleq
\mbox{map } \left( \lambda x . x^2 \right) \mbox{ items}
\caption{{\ell}^2 using map}
\label{eq:l2_map}
\end{equation}
\newline

              \schemelist{../chapter-2/sicp_ch2_e2-21.scm}
              \outlist{../output/sicp_ch2_e2-21.out}

Exercise 2.23. The procedure for-each is similar to map. It takes as arguments a procedure and a list of elements. However, rather than forming a list of the results, for-each just applies the procedure to each of the elements in turn, from left to right. The values returned by applying the procedure to the elements are not used at all -- for-each is used with procedures that perform an action, such as printing. For example,
\newline

\begin{verbatim}
  (for-each (lambda (x) (newline) (display x))
    (list 57 321 88))

  57
  321
  88
\end{verbatim}
\newline

The value returned by the call to for-each (not illustra above) can be something arbitrary, such as true. Give an implementation of for-each. 
\newline

\begin{equation}
\mbox{for-each}\left(f,x\right) \triangleq
\begin{cases}
\mobx{\#t} & \mobx{null?\ x} \\
\mbox{progn: } f x_A \mbox{\ ;\ } \mbox{for-each}\left(f,x_D\right) & \mbox{o.w.} \\
\end{cases}
\label{eq:foreach}
\end{equation}
\newline

              \schemelist{../chapter-2/sicp_ch2_e2-23.scm}
              \outlist{../output/sicp_ch2_e2-23.out}

Suppose we evaluate the expression (list 1 (list 2 (list 3 4))).  Give the result printed by the interpreter, the corresponding box-and-pointer structure, and the interpretation of this as a tree (as in figure 2.6). 

\begin{figure}[H]
\begin{center}
%\resizebox{7in}{!}{\includegraphics{./graphviz/e-2-24-box-and-pointer.png}}
\includegraphics{./graphviz/e-2-24-box-and-pointer.png}
\end{center}
\caption{box and pointer representation}
\label{fig:box_and_pointer}
\end{figure}

\begin{figure}[H]
\begin{center}
%\resizebox{7in}{!}{\includegraphics{./graphviz/e-2-24-binary.png}}
\includegraphics{./graphviz/e-2-24-binary.png}
\end{center}
\caption{tree representation}
\label{fig:binary}
\end{figure}

              \schemelist{../chapter-2/sicp_ch2_e2-24.scm}
              \outlist{../output/sicp_ch2_e2-24.out}

Exercise 2.25. Give combinations of cars and cdrs that will pick 7 from each of the following lists:

\begin{verbatim}
  L1: (1 3 (5 7) 9)
  L2: ((7))
  L3: (1 (2 (3 (4 (5 (6 7))))))
\end{verbatim}
\newline

              \schemelist{../chapter-2/sicp_ch2_e2-25.scm}
              \outlist{../output/sicp_ch2_e2-25.out}

            \subsubsection{Sequences as Conventional Interfaces}
            \subsubsection{Example: A Picture Language}
Exercise 2.44.  Define the procedure up-split used by corner-split.  It is similar to right-split, except that it switches the roles of below and beside.
              \schemelist{../chapter-2/sicp_ch2_e2-44.scm}
              \outlist{../output/sicp_ch2_e2-44.out}

\begin{figure}[H]
\begin{center}
%\resizebox{7in}{!}{\includegraphics{../figures/sicp_ch2_e2-44.png}}
\includegraphics{../figures/sicp_ch2_e2-44.png}
\end{center}
\caption{Up Split 2}
\label{fig:up_split_2}
\end{figure}

Exercise 2.45.  Right-split and up-split can be expressed as instances of a general splitting operation. Define a procedure split with the property that evaluating
\begin{verbatim}
  (define right-split (split beside below))
  (define up-split (split below beside))
\end{verbatim}
produces procedures right-split and up-split with the same behaviors as the ones already defined.
              \schemelist{../chapter-2/sicp_ch2_e2-45.scm}
              \outlist{../output/sicp_ch2_e2-45.out}

\begin{figure}[H]
\begin{center}
%\resizebox{7in}{!}{\includegraphics{../figures/sicp_ch2_e2-45_right.png}}
\includegraphics{../figures/sicp_ch2_e2-45_right.png}
\end{center}
\caption{Right Split 2}
\label{fig:right_split_2}
\end{figure}

\begin{figure}[H]
\begin{center}
%\resizebox{7in}{!}{\includegraphics{../figures/sicp_ch2_e2-45_up.png}}
\includegraphics{../figures/sicp_ch2_e2-45_up.png}
\end{center}
\caption{Up Split 2}
\label{fig:up_split_2}
\end{figure}

Exercise 2.46.  A two-dimensional vector v running from the origin to a point can be represented as a pair consisting of an x-coordinate and a y-coordinate. Implement a data abstraction for vectors by giving a constructor make-vect and corresponding selectors xcor-vect and ycor-vect. In terms of your selectors and constructor, implement procedures add-vect, sub-vect, and scale-vect that perform the operations vector addition, vector subtraction, and multiplying a vector by a scalar:
              \schemelist{../chapter-2/sicp_ch2_e2-46.scm}
              \outlist{../output/sicp_ch2_e2-46.out}

        \subsection{Symbolic Data}
            \subsubsection{Quotation}

Exercise 2.54. Two lists are said to be equal? if they contain equal 
elements arranged in the same order. For example,
\newline

(equal? '(this is a list) '(this is a list))
\newline

is true, but
\newline

(equal? '(this is a list) '(this (is a) list))
\newline

is false. To be more precise, we can define equal? recursively in terms of 
\newline

the basic eq? equality of symbols by saying that a and b are equal? if 
they are both symbols and the symbols are eq?, or if they are both lists 
such that (car a) is equal? to (car b) and (cdr a) is equal? to (cdr b). 
Using this idea, implement equal? as a procedure. 
\newline

Comparing to structures using $equals?$:
\newline

\begin{equation}
\left(equals?\mbox{ a b}\right) = 
f\left(a,b\right) = 
\begin{cases}
\bot                                                  & \mbox{if } S\left(a\right) \oplus S\left(b\right) \\
a \stackrel{?}{=} b                                   & \mbox{if } S\left(a\right) \wedge S\left(b\right) \\
f\left(a_A,b_A\right) \lwedge f\left(a_D,b_D\right)   & \mbox{if } L\left(a\right) \wedge L\left(b\right)
\end{cases}
\label{eq:def_equals}
\end{equation}
\newline

\text{where}
\newline

\begin{equation}
L\left(a\right) \wedge L\left(b\right)
\Longrightarrow
\lnot \left( S\left(a\right) \wedge S\left(b\right) \right)
\end{cases}
\label{eq:def_equals_membership}
\end{equation}


To show that all cases have been exhausted (\ref{eq:def_equals_table}):
\newline

\begin{equation}
\begin{array}{cc|ccc}
S\left(a\right) & 
S\left(b\right) & 
S\left(a\right) \wedge S\left(b\right) & 
S\left(a\right) \oplus S\left(b\right) & 
\lnot \left( S\left(a\right) \wedge S\left(b\right) \right)
\\
\hline
F & F & F                   & F                       & \textcolor{blue}{T} \\
F & T & F                   &     \textcolor{blue}{T} & F                   \\
T & F & F                   &     \textcolor{blue}{T} & F                   \\
T & T & \textcolor{blue}{T} & F                       & F                   \\
\end{array}
\label{eq:def_equals_table}
\end{equation}

At this point, the recursive form is functional, but it is not expressed in tail-recursive form, and as such is not subject to tail-call optimization.  The following is a conversion to tail-recursive form:
\newline
\begin{equation}

\begin{equation}
\left(equals?\mbox{ a b}\right) = 
f\left(a,b\right) = f_k\left(a,b,\top\right)
\label{eq:def_equals_tco}
\end{equation}
\newline

\begin{equation}
f_k\left(a,b,p\right) = 
\begin{cases}
\bot                                                & \mbox{if } S\left(a\right) \oplus S\left(b\right) \\
p \wedge \left( a \stackrel{?}{=} b \right)         & \mbox{if } S\left(a\right) \wedge S\left(b\right) \\
f_k\left(a_D,b_D, f_k\left(a_A,b_A,p\right)\right)  & \mbox{otherwise}
\end{cases}
\label{eq:def_equals_tco_iter}
\end{equation}
\newline

              \schemelist{../chapter-2/sicp_ch2_e2-54.scm}
              \outlist{../output/sicp_ch2_e2-54.out}
Exercise 2.55. Eva Lu Ator types to the interpreter the expression (car 'abracadabra)
\newline
To her surprise, the interpreter prints back quote. Explain.
              \schemelist{../chapter-2/sicp_ch2_e2-55.scm}
              \outlist{../output/sicp_ch2_e2-55.out}
            \subsubsection{Example: Symbolic Differentiation}
Exercise 2.56.  Show how to extend the basic differentiator to handle more kinds of expressions. For instance, implement the differentiation rule
\begin{equation}
\frac{\partial d \left( u^n \right)}{\partial u}
= n u^{-1} \frac{\partial u}{\partial x}
\label{eq:deriv_exp_2}
\end{equation}
\newline
              \schemelist{../chapter-2/sicp_ch2_e2-56.scm}
              \outlist{../output/sicp_ch2_e2-56.out}

Exercise 2.73. Section 2.3.2 described a program that performs symbolic differentiation:
\newline

\begin{verbatim}
 (define (deriv exp var)
   (cond ((number? exp) 0)
     ((variable? exp) (if (same-variable? exp var) 1 0))
     ((sum? exp)
       (make-sum (deriv (addend exp) var)
                 (deriv (augend exp) var)))
       ((product? exp)
          (make-sum
          (make-product (multiplier exp)
                        (deriv (multiplicand exp) var))
          (make-product (deriv (multiplier exp) var)
                        (multiplicand exp))))
 
  <more rules can be added here>
 
       (else (error "unknown expression type -- DERIV" exp))))
\end{verbatim}


We can regard this program as performing a dispatch on the type of the expression to be differentiated.  In this situation the "type tag" of the datum is the algebraic operator symbol (such as +) and the operation being performed is deriv. We can transform this program into data-directed style by rewriting the basic derivative procedure as 
\newline

\begin{verbatim}
  (define (deriv exp var)
   (cond ((number? exp) 0)
     ((variable? exp) (if (same-variable? exp var) 1 0))
     (else ((get 'deriv (operator exp)) (operands exp)
       var))))
  (define (operator exp) (car exp))
  (define (operands exp) (cdr exp))
\end{verbatim}

Constant Rule
\begin{equation}
\frac{\partial c}{\partial x} = 0
\label{eq:constant_rule}
\end{equation}
\newline

Sum Rule
\begin{equation}
\frac{\partial}{\partial x} \left( u + v \right) = \frac{\partial u}{\partial x} + \frac{\partial v}{\partial x} 
\label{eq:product_rule}
\end{equation}
\newline

Product Rule
\begin{equation}
%\frac{d}{dx} \left( f \cdot g \right) = f \cdot g^{'} + f^{'} \cdot g = v \cdot \frac{du}{dx} + u \cdot \frac{dv}{dx} 
\frac{\partial}{\partial x} \left( f \cdot g \right) = v \cdot \frac{\partial u}{\partial x} + u \cdot \frac{\partial v}{\partial x} 
\label{eq:product_rule}
\end{equation}
\newline

Exponent Rule
\begin{equation}
\frac{\partial d \left( u^n \right)}{\partial u} = n u^{-1} \frac{\partial u}{\partial x}
\label{eq:deriv_exp}
\end{equation}
\newline

% Quotient Rule
% \begin{equation}
% \frac{d}{dx} \frac{f}{g} = \frac{(f^{'} \cdot g - g^{'} \cdot f )}{g^2}
% \label{eq:quotient_rule}
% \end{equation}
% \newline

% Reciprocal Rule 
% \begin{equation}
% \frac{d}{dx} \frac{1}{f} = \frac{f^{'}}{f^2}
% \label{eq:reciprocal_rule}
% \end{equation}
% \newline

Chain Rule
\begin{equation}
%\frac{d}{dx} \left( f \circ g \right) = (f^{'} \circ g) \cdot g^{'}
\frac{\partial}{\partial x} \left( f \circ g \right) = (\frac{\partial f}{\partial x} \circ g) \cdot \frac{\partial g}{\partial x}
\label{eq:chain_rule}
\end{equation}
\newline

a. Explain what was done above. Why can't we assimilate the predicates number? and same-variable? into the data-directed dispatch?
\newline

b. Write the procedures for derivatives of sums and products, and the auxiliary code required to install them in the table used by the program above.
\newline

c. Choose any additional differentiation rule that you like, such as the one for exponents (exercise 2.56), and install it in this data-directed system.
\newline

d. In this simple algebraic manipulator the type of an expression is the algebraic operator that binds it together. Suppose, however, we indexed the procedures in the opposite way, so that the dispatch line in deriv looked like
\newline

\begin{verbatim}
  ((get (operator exp) 'deriv) (operands exp) var)
\end{verbatim}

What corresponding changes to the derivative system are required? 
\newline

              \schemelist{../chapter-2/sicp_ch2_e2-73.scm}
              \outlist{../output/sicp_ch2_e2-73.out}

            \subsubsection{Example: Representing Sets}
            \subsubsection{Example: Huffman Encoding Trees}
        \subsection{Multiple Representations for Abstract Data}
            \subsubsection{Representations for Complex Numbers}
            \subsubsection{Tagged data}
            \subsubsection{Data-Directed Programming and Additivity}
        \subsection{Systems with Generic Operations}
            \subsubsection{Generic Arithmetic Operations}
            \subsubsection{Combining Data of Different Types}
            \subsubsection{Example: Symbolic Algebra}

%% *EOF*
